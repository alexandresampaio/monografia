\chapter{SGBDS}
\thispagestyle{empty}

\section{Breve Histórico}
Atualmente o mundo enfrenta uma quantidade siguinificativa mediante a demanda de dados trafegando simultaneamente. E desde sua criação
em \cite{COOD}, o modelo Relacional tem sido abrangentemente utilizado ao redor do mundo como fonte de dados armazenado em servidores de dados para diversas
finalidades. Esse sistema de gerenciamento de banco dados mundialmente conhecido como SGBDs, tais como Mysql, PostgreSQL, Oracle, SQLServer entre outros
foram sendo utilizados ao percurso de muitos anos, sendo adotados como padrão por muitas empresas desde comuns até aquelas funcionam apenas como desenvolvedoras
de \textbf{software}, atendendo comletamente o propósito de cada uma com suas diferentes aplicações, contudo com o avanço das tecnologias empregadas no desenvolvimento
de sistemsas web no \textit{século XXI}, com o grande aumento da demanda de dados e com o advento do surgimento das redes sociais, grandes portais com número, intermitente de 
informações a todo momento, sites de compra coletiva entre outros tipos de sistemas, a arquitetura dos \textbf{SGBDs} começou a ser questionada por apresentar certas limitações 
quanto a escalabilidade e performance.

A escalabilidade de um SGBD está relacionada à capacidade de um software crescer de tal forma rápida e intuitiva, que possa atender a uma demanda cada vez maior de 
dados trafegando. A performance do software que faz uso de um SGBDs, por sua vez, faz referenĉia ao tempo de resposta das requisições efetuadas pelo usuário que faz uso desse sistema.

Tendo em vista, que o surgimento dessas novas aplicações fizeram com que a indústria de de desenvolvimento de software apresentasse novas idéias expecificas para estas duas situações concistentes
existentes devido às limitações dos sistemas de armazenamento de dados relacionais. As novas aplicações web exploram conteúdo em abrangencia na Web 2.0 e diante dos problemas encontrados, foram 
desenvolvidas novas soluções proprietárias \cite{CUNHA} que diferem do paradigma relacional, as quais foram todas reunidas sob o nomenclatura \textit{NoSQL (Not Only SQL)}.

Este capitulo procura estabelecer uma além dessa breve introdução um comparativo entre banco de dados relacionais e os NoSQL.

\section{Banco de Dados Relacionais}

\section{NoSQL}

\section{Comparativo entre SGBDS Relacionais e Não Relacionais}
