\chapter{Banco de Dados - SGBDS x NoSQL}
\thispagestyle{empty}
Este capitulo procura estabelecer além de uma breve introdução terminológica, um comparativo entre banco de dados relacionais e os NoSQL.

\section{Breve Histórico}
Atualmente o mundo enfrenta uma quantidade siguinificativa, mediante a demanda de dados trafegados simultaneamente pela rede. Desde o surgimento do modelo
relacional de armazenamento de dados por \cite{CODD}, este tem sido abrangentemente utilizado ao redor do mundo para diversas finalidades. Esse sistema de 
gerenciamento de banco dados \textbf{mundialmente} conhecido como SGBDs, tais como Mysql, PostgreSQL, Oracle, SQLServer entre outros foram sendo utilizados ao 
percurso de muitos anos, sendo estes adotados como padrão por muitas empresas desde comuns até desenvolvedoras de \textbf{software}, atendendo comletamente ao
propósito de cada uma com suas diferentes situações e aplicações, contudo com o avanço das tecnologias empregadas no ramo do desenvolvimento de sistemsas web 
no \textit{século XXI}, com o grande aumento da demanda de dados e com o advento do surgimento das redes sociais, grandes portais com número, intermitente de
informações a todo momento, sites de compra coletiva entre outros tipos de sistemas, a arquitetura dos \textbf{SGBDs} começou a ser questionada por apresentar 
certas limitações quanto a escalabilidade e performance. A escalabilidade de um SGBD está relacionada à capacidade de um software crescer de tal forma, rápida
e intuitiva e eficiente, que possa atender a uma demanda cada vez maior de dados. A performance do software que faz uso de um SGBDs, por sua vez, faz referência
à estimativa do tempo de resposta das requisições efetuadas pelo usuário que faz uso de determinado sistema.

Tendo em vista, o surgimento dessas novas aplicações, fez com que a indústria de de desenvolvimento de software apresentasse novas idéias expecificas para estas
duas situações concistentes e existentes devido às limitações dos sistemas de armazenamento de dados relacionais. As novas aplicações web exploram conteúdo em 
abrangencia na Web 2.0 e diante dos problemas encontrados, foram desenvolvidas novas soluções proprietárias, que diferem do paradigma relacional, 
as quais foram todas reunidas sob a nomenclatura \textit{NoSQL (Not Only SQL)}.\cite{CUNHA}.


\section{SGBDs Relacionais}
Existe uma ampla diferença entre Dados, Informação e Conhecimento.
\begin{itemize}
  \item{ \textbf{ Dado: } Qualquer fato, instrução formalizada para uso da comunicação, um código para uma obra-prima de objetos, ou seja é a informação não tratada.
  
    \centering{ \textit{"Dados são itens referentes a uma descrição primária de objetos, eventos, atividades e transações 
    que são gravados, classificados e armazenados, mas não chegam a ser organizados de forma a transmitir 
    algum significado específico."}\cite{MCLEAN_WETHERBE}.}
  }

  \item{ \textbf{ Informação: } É a coleção de tudo aquilo que faz sentido e possui valor constitucional e siguinificante aquelas às quais pode se tirar algum proveito para determinado propósito.
  
    \centering{ \textit{"Informação é todo conjunto de dados organizados de forma a terem sentido e valor para seu destinatário. 
      Este interpreta o significado, tira conclusões e faz deduções a partir deles"}\cite{MCLEAN_WETHERBE}. } 
  }
  
  \item{ \textbf{ Conhecimento: } Assim como conhecimentto É a coleção de tudo aquilo que faz sentido e possui valor constitucional e siguinificante 
  aquelas às quais pode se tirar algum proveito para determinado propósito, conhecimento é a junção de todos os dados e informações processados afim de transmitir
  a compreensão, experiência, aprendizado acumulativo e alguma técnica, quando aplicada a determinada situação ou atividade que dependa de tal conhecimento. 
  
    \centering{ \textit{"Conhecimento consiste de dados e informações organizados e processados para
    transmitir compreensão, experiência, aprendizado acumulado e técnica, quando se aplicam a determinado problema ou atividade"}\cite{MCLEAN_WETHERBE}. } 
  }
\end{itemize}

Um das grandes e principais características do modelo de dados relacionais são as restrições de integridade:



\section{ACID}
  \begin{itemize}
    \item{ A - Atomicidade:
      Todas as operações da trançações serão executadas por completo, caso contrario não será executada em momento algum.
    }
    
    \item{ C - Concistência:
      
    }
  \end{itemize}

\section{NoSQL}

\section{Comparativo entre SGBDS Relacionais e Não Relacionais}
