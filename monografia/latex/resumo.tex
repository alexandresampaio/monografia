\begin{center}
\textbf{RESUMO}
\end{center}
\singlespacing

\noindent Este trabalho descreve sobre o desenvolvimento de um sistema de controle de atividades complementares imposto pelo Instituto Federal Fluminense, 
criado sob a linguagem de programação Ruby com o auxílio de seu web framework o Rails com intuito de realizar um desenvolvimento ágil com auxilio de 
Behavior Driven Development. Para concretizar esse trabalho durante todo o processo de desenvolvimento e de test foram utilizado varios conceitos de Ruby. 
Sendo o objetivo desta ferramenta é auxiliar tanto o aluno bem como um professor avaliador a computar notas para estas atividades. Foi-se utilizado um sistema 
a base UNIX, e um ambiente de desenvolvimento para criar a aplicação, bem como base de dados para se criar todo o procedimento. 
Foram utilizadas as bases Mysql e MongoDB com comprativo entre bancos de dados relacionais e não relacionais.

\noindent PALAVRAS-CHAVE:  Serviço Web, Software livre, UNIX,  Ruby, Rails, Mysql, NoSQL, MongoDB 
